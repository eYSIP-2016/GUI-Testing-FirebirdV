\documentclass[a4paper,12pt,oneside]{book}

%-------------------------------Start of the Preable------------------------------------------------
\usepackage[english]{babel}
\usepackage{blindtext}
%packagr for hyperlinks
\usepackage{hyperref}
\hypersetup{
    colorlinks=true,
    linkcolor=blue,
    filecolor=magenta,      
    urlcolor=cyan,
}

\urlstyle{same}
%use of package fancy header
\usepackage{fancyhdr}
\setlength\headheight{26pt}
\fancyhf{}
%\rhead{\includegraphics[width=1cm]{logo}}
\lhead{\rightmark}
\rhead{\includegraphics[width=1cm]{logo}}
\fancyfoot[RE, RO]{\thepage}
\fancyfoot[CE, CO]{\href{http://www.e-yantra.org}{www.e-yantra.org}}

\pagestyle{fancy}

%use of package for section title formatting
\usepackage{titlesec}
\titleformat{\chapter}
  {\Large\bfseries} % format
  {}                % label
  {0pt}             % sep
  {\huge}           % before-code
 
%use of package tcolorbox for colorful textbox
\usepackage[most]{tcolorbox}
\tcbset{colback=cyan!5!white,colframe=cyan!75!black,halign title = flush center}

\newtcolorbox{mybox}[1]{colback=cyan!5!white,
colframe=cyan!75!black,fonttitle=\bfseries,
title=\textbf{\Large{#1}}}

%use of package marginnote for notes in margin
\usepackage{marginnote}

%use of packgage watermark for pages
%\usepackage{draftwatermark}
%\SetWatermarkText{\includegraphics{logo}}
\usepackage[scale=2,opacity=0.1,angle=0]{background}
\backgroundsetup{
contents={\includegraphics{logo}}
}

%use of newcommand for keywords color
\usepackage{xcolor}
\newcommand{\keyword}[1]{\textcolor{red}{\textbf{#1}}}

%package for inserting pictures
\usepackage{graphicx}

%package for highlighting
\usepackage{color,soul}

%new command for table
\newcommand{\head}[1]{\textnormal{\textbf{#1}}}


%----------------------End of the Preamble---------------------------------------


\begin{document}

%---------------------Title Page------------------------------------------------
\begin{titlepage}
\raggedright
{\Large eYSIP2016\\[1cm]}
{\Huge\scshape GUI DEVELOPMENT FOR FIREBIRD V USING JAVA \\[.1in]}
\vfill
\begin{flushright}
{\large INTERN: Jatin Mittal\\}
{\large MENTORS: Yogita Mali \\}
{\large Abhishek Singh \\}
{\large Sachin Gupta \\}
{\large Duration of Internship: $ 21/05/2016-10/07/2016 $ \\}
\end{flushright}

{\itshape 2016, e-Yantra Publication}
\end{titlepage}
%-------------------------------------------------------------------------------

\chapter[Project Tag]{GUI development for Firebird V using Java}
\section*{Abstract}
	In this project, a Graphical User Interface(GUI) is designed which is platform independent i.e it can run on all the Operating Systems Windows, Linux, Mac. JAVA language is used to design the GUI to make it platform independent. The GUI is used to test all the components of Firebird V ATmega2560 robot. GUI gives the reading of all the sensors and battery voltage, controls the motion and velocity of robot, rotate the robot by some angle and move forward and backward, set the velocities of both the motor, rotate the servo motor, prints on LCD and glows the Bar Graph LED.
\newpage
\subsection*{Objective}
	\begin{enumerate}
		\item [$\bullet$] Develop a GUI using JAVA AWT and Swing for testing different Firebird Components.\\
		\item [$\bullet$] GUI should communicate with Firebird using Serial Communication.\\
		\item [$\bullet$] GUI must also actuate DC motors, along with velocity control and servo motor.
		\item [$\bullet$] GUI should be compatible with all the platforms. 
	\end{enumerate}
	\vspace{1cm}
		\begin{center}
	\begin{tabular}{|c|p{2.5in}|c|}
		\hline
		\textbf{Task No.} & \qquad \qquad \qquad \qquad \textbf{Task} & \textbf{Deadline}\\
		\hline
		$1$ & Installation and Study JAVA- AWT, Swing Layout Manager & 2 Days\\
		\hline
		$2$ & JAVA Serial Communication & 2 Days\\
		\hline
		$3$ & Study and test the existing GUI & 2 Days\\
		\hline
		$4$ & Make GUI platform independent and with resiable window & 4 Days\\
		\hline
		$5$ & Add button to download firmware hex file from the GUI & 3 Days\\
		\hline
		$6$ & Complete testing of the GUI & 4 Days\\
		\hline
		$7$ & Creating executable file and Documentation & 5 Days\\
		\hline
	\end{tabular}
	\end{center}
\newpage
\subsection*{Completion status}
\begin{enumerate}
	 	\item \textbf{\large Installation and Study JAVA:}
	 	\begin{itemize}
			\item Installed Netbeans IDE in my computer to develop the JAVA code.
			\item Learned about JAVA AWT, Swing and Layout Manager:
			\begin{itemize}
				\item Java AWT (Abstract Windowing Toolkit) is an API to develop GUI or window-based application in java. Java AWT components are platform-dependent i.e. components are displayed according to the view of operating system. AWT is heavyweight i.e. its components uses the resources of system. The java.awt package provides classes for AWT api such as TextField, Label, TextArea, RadioButton, CheckBox, Choice, List etc.
				\item Swing- Java Swing is used to create window-based applications. It is built on the top of AWT (Abstract Windowing Toolkit) API and entirely written in java. Unlike AWT, Java Swing provides platform-independent and lightweight components. The javax.swing package provides classes for java swing API such as JButton, JTextField, JTextArea, JRadioButton, JCheckbox, JMenu, JColorChooser etc.
				
				\item Layout Manager- The LayoutManagers are used to arrange components in a particular manner. LayoutManager is an interface that is implemented by all the classes of layout managers. There are following classes that represent the layout managers: BorderLayout, FlowLayout, GridLayout, CardLayout, GridBagLayout, BoxLayout, GroupLayout, ScrollPaneLayout, SpringLayout etc.
			\end{itemize}
		\end{itemize}
		\item \textbf{\large Study JAVA Serial Communication} 
		\begin{itemize}
			\item Used RxTx Library for Serial Communication.
			\item Developed the code to search for all the available COM ports. Searched for COM port using getPortIdentifiers() method of CommPortIdentifier class. It creates an enumeration object of all the ports and stores it in an ArrayList and then converts it into String array and add it in JComboBox. 
		    \item Developed the code to connect to the serial port by opening the port and set its parameters such as baudrate, data bits, stop bits, parity bit and setting the FLOWCONTROL. \\
		    Code to connect the serial port is as follows:
		    \newpage
		    \begin{flushleft}
		    	\lstset{language=Java, showspaces=false,
		    		showstringspaces=false, tabsize=1, breaklines=true}
		    	\begin{lstlisting}
		    	port = portId.open("Demo Application", 5000); 
		    	serialport = (SerialPort)port; 
		    	int baudRate = 115200; 
		    	serialport.setSerialPortParams( 
		    	                   baudRate, 
		    	                   SerialPort.DATABITS_8, 
		    	                   SerialPort.STOPBITS_1, 
		    	                   SerialPort.PARITY_NONE); 
		    	serialport.setFlowControlMode(SerialPort.FLOWCONTROL_NONE); 
		    	\end{lstlisting}
		    \end{flushleft}
		    The above code throws UnsupportedCommOperationException, PortInUseException and NoSuchPortException exception.
			\item Developed the code to disconnect the serial port by closing the port. To disconnect the port close the serial port, output stream and input stream and removes event listener as follows: 
			\begin{flushleft}
				\lstset{language=Java, showspaces=false,
					showstringspaces=false, tabsize=1, breaklines=true}
				\begin{lstlisting}
					serialport.removeEventListener(); 
					serialport.close(); 
					outputstream.close(); 
					inputstream.close();
				\end{lstlisting}
			\end{flushleft}
			\item Developed the code to write on serial port using output stream. write(byte[]) method of OutputStream class is used to witre any string on the serial port.Code to write on serial port is as follow- 
			\begin{flushleft}
				\lstset{language=Java, showspaces=false,
					showstringspaces=false, tabsize=1, breaklines=true}
				\begin{lstlisting}
				outputstream.write(serialmessage.getBytes()); 
				outputstream.flush();
				\end{lstlisting}
			\end{flushleft}
			\item Developed the code to read from serial port by using serial event listener and using the input stream. In this a class is created implementing the interface SerialPortEventListener whose serialEvent method is called whenever data is available in inputstream.  
		\end{itemize}
		\newpage
		\item \textbf{\large Develop GUI for Various Components}
		\begin{itemize}
		    	\item \textbf{Created Section to download Firmware hex file in GUI:}
			\begin{figure}[h]
				\begin{center}
					\includegraphics[scale=0.75]{browse.png}
				\end{center}
			\end{figure}
			\begin{itemize}
				\item Added a panel in Jframe for downloading firmware hex file in the Firebird V robot.
				\item Created Browse button to select the .hex file from the system to be downloaded.
				\item  Created a TextField which displays the complete path of the selected file.
				\item Created a Program button to proceed the firmware downloading.
			\end{itemize}
			
			\item \textbf{Created COM PORT Section in GUI:}
			\begin{figure}[h]
				\begin{center}
					\includegraphics[scale=1]{comport.png}
				\end{center}
			\end{figure}
			\begin{itemize}
				\item Added a panel in Jframe for COM port section and then added jComboBox showing all the available COM ports.
				\item Then added two buttons one to connect to the robot and other to disconnect from the robot.
				\item So after selecting the COM port and clicking on the connect button, the GUI connects to the robot and now reading from the serial port and writing on the serial port is possible
			\end{itemize}
			\newpage
			\item \textbf{Created Buzzer Section in GUI:}
			\begin{figure}[h]
				\begin{center}
					\includegraphics[scale=1]{buzzer.png}
				\end{center}
			\end{figure}
			\begin{itemize}
				\item Created a panel for buzzer section in main JFrame and added a label giving the section title and a button to turn the buzzer on and off in the panel.
				\item To turn on the buzzer "7" or 0x37 is written on the serial port and to turn off the buzzer "9" or 0x39 is written.   
			\end{itemize}   
			\newpage
			\item \textbf{Created Motion Control Section in GUI:}
			\begin{itemize}
				\item Here's the panel added for motion control.\\
				\begin{figure}[h]
					\begin{center}
						\includegraphics[scale=1]{motioncontrol.png}
					\end{center}
				\end{figure}
				\item  Controls the forward, backward, right, left and stop the motion.
				\item On pressing the forward button "8" which is 0x38 in hex is written on the output stream of serial port and firebird stores this value in UDR2 register and performs the forward action corresponding to that value. 
				\item Similarly for backward motion "2" or 0x32 in hex, for left motion "6" or 0x36 in hex, for right "4" or 0x34 and for stop "5" or 0x35 in hex is written on the output stream of serial port to perform the corresponding action.
			\end{itemize}  
			\newpage    
			\item \textbf{Created Velocity Control Section in GUI:}
			\begin{itemize}
				\item There are two sliders to set the velocity of both left and right motor. After setting the velocity bot will perform the motion with specified velocities.\\
				\begin{figure}[h]
					\begin{center}
						\includegraphics[scale=1]{velocitycontrol.png}
					\end{center}
				\end{figure}
				\item On pressing the set button 0x52, left motor velocity and right motor velocity in hex is written on the output stream of serial port. In the firebird V robot on getting the 0x52 in UDR2 register next two values are stored as left and right velocities.
				\item On clicking the Set button, if both velocities are selected then a dialog box displaying the message that both the velocities have been set will be displayed and if any one velocity is not selected then dialog box showing the message to set the velocities will be displayed.
			\end{itemize} 
			\item \textbf{Created Section for Rotation by some angle and Movement by some Distance in GUI:} \\
			\begin{itemize}
				\item Added panel for Rotation and Movement of the bot and added two text boxes to enter the value of angle and distance and also added button to rotate left and right by that angle and move forward and backward by that distance.
			
				\begin{figure}[h]
					\begin{center}
						\includegraphics[scale=1]{rotation_movement.png}
					\end{center}
				\end{figure} 
				\item User have to give the angle to rotate it left or right and have to give distance in millimeter to move that distance forward and backward.
				\item To get the value of angle to rotate and distance to move divide the value by 255 and take the mod of value with 255 and send both the value to the bot along with the some hex value which is used to identify which action to perform.
				\item For left rotation 0x57, for right rotation 0x57, for forward movement 0x55 and for backward movement 0x56 is written on the output stream of serial port.
			\end{itemize}   
			\item \textbf{Developed GUI to give the Reading of Sensors: } 
			\begin{itemize}
				\item There are 3 sections of sensors first section for White Line Sensor readings, second section for Distance Sensor readings and third for IR Sensor readings.
				\item  Hex value of ASCII character "T" i.e 0x54 is send to bot to get the reading of all the white line sensors, IR sensors, Distance sensors and battery voltage. On writing the 0x54 on serial port bot sends all the values to the GUI and then GUI displays the sensors reading  using progress bar.
				\item White Line Sensors reading look like this: \\
				\begin{figure}[h]
					\begin{center}
						\includegraphics[scale=1]{whitelinesensor.png}
					\end{center}
				\end{figure}
				The White Line Sensors reading varies from 0 to 255 and on white surface sensors give lower reading and on black surface sensors give higher reading.
				
				\item IR Sensors reading look like this: \\ 
				\begin{figure}[h]
					\begin{center}
						\includegraphics[scale=0.75]{irsensor.png}
					\end{center} 
				\end{figure} \\
				IR Sensors 6, 7 and 8 works only when jumper 4 is connected. Sometimes IR sensors reading of 6, 7 and 8 on progress bar does not match the reading on the console. Like reading of these sensors is 0,0,0 on console but on progress bar it shows 90,0,0. IR Sensors reading varies from 0 to 255.
				\newpage
				\item Distance Sensors reading look like this: \\
				\begin{figure}[h]
					\begin{center}
						\includegraphics[scale=0.75]{distance_sensor.png}
					\end{center}
				\end{figure} \\
				This is reading when 3rd distance sensor is connected. All other readings are garbage value. Distance Sensors reading varies from 0 mm to 800 mm.
			\end{itemize} 
			\newpage
			\item \textbf{Created Battery Voltage Section in the GUI:}\\
			 This section gives the battery voltage reading. Maximum value of battery voltage is 12V.
				\begin{figure}[h]
			 		\begin{center}
			 			\includegraphics[scale=0.75]{batteryvoltage.png}
			 		\end{center}
			 	\end{figure} 
			 \item \textbf{Created Section For Servo Motor in GUI:} \\
			 \begin{itemize}
			 	\item Added panel in JFrame for servo motors section. There are three sliders and text boxes to set the angle of servo motor to be rotated.
			 	
			 	\begin{figure}[h]
			 		\begin{center}
			 			\includegraphics[scale=0.75]{servomotor.png}
			 		\end{center}
			 	\end{figure} 
			 	\item Rotates the Servo Motor S1, S2, S3 by the angle given by user using slider or text box. Servo Motor can rotate form 0 to 180 degrees.
			 	\item Servo motor first goes to its initial position and then rotate to the specified angle.
			 	\item On clicking the rotate button for servo motor 1 0x80 is written on serial port. Similarly 0x 81 for servo motor 2 and 0x82 for servo motor 3 indicating that next value received by the bot is angle to be rotate.
			 	\item Servo Motor should be improved such that it sets the specified angle without going to its initial position. 
			 \end{itemize}
			 \item \textbf{Created Section to Print on LCD in GUI:}
			 \begin{itemize}
			 	\item User can print one character on the LCD by giving the row, column and character to print. \\
			 	\begin{figure}[h]
			 		\begin{center}
			 			\includegraphics[scale=1]{lcd.png}
			 		\end{center}
			 	\end{figure}
			 	\item On clicking the print button sends 0x83 to identify that next three values which the bot gets are row, column and text to be printed on LCD. 
			 \end{itemize}
			 \newpage 
		   	 \item \textbf{Created Section for Bar Graph LED in GUI:} 
		   	 \begin{itemize}
		   	 	\item User can glow Bar Graph LED by giving the LED number. On clicking the glow button 0x84 is written on the serial port to identify next value which is written is the LED to glow.\\
		   	 	\begin{figure}[h]
		   	 		\begin{center}
		   	 			\includegraphics[scale=1]{bargraphled.png}
		   	 		\end{center}
		   	 	\end{figure}
		   	 \end{itemize} 
		   	 \newpage
			 
			 \item This is how final GUI looks like- \\
			  Using various sections of this GUI test all the components of Firebird V robot.
			 \begin{figure}[h]
				\begin{center}
					\includegraphics[scale=0.60]{GUI.png}
				\end{center}
			 \end{figure}
			  
		\end{itemize}
		\item \textbf{\large Creating Executable File and Running the GUI on Different Systems:} \\
		\begin{itemize}
			\item Currently we can run the .jar file in command prompt and explicitly copying the rxtxSerial.dll file. Copy rxtxSerail.dll in \%JAVAHOME\% jdk/jre/bin. And then give the command "java -jar filename.jar" in command prompt.
		    \item The GUI is running on 32-bit and 64-bit windows 7, Linux and Windows 8.1 OS. Testing the GUI Mac is remaining.
			\item RxTx Library is not supported in Windows 8 OS.
		\end{itemize} 
		\item \large \textbf{Pending Tasks:}
		\begin{itemize}
			\item Button to download firmware hex file from the GUI.
			\item Creating executable file.
		\end{itemize}
		\newpage
		\item \textbf{\large Hardware used:}
		\begin{itemize}
		    \item Firebird V Robot 
            \item 
            \href[page=5]{./datasheet/AtMega2560.pdf}{Firebird V Robot - Datasheet}
            \item
            \href{http://www.nex-robotics.com/products/fire-bird-v-robots/fire-bird-v-atmega2560-robotic-research-platform.html}{Firebird V Robot - Vendor link}
        \end{itemize}
        
        \item \textbf{\large Software used:}
        \begin{itemize}
            \item NetBeans IDE 8.1 - 
             \href{https://netbeans.org/downloads/}{Download}
             \item Java SE Development Kit 8 - 
             \href{http://www.oracle.com/technetwork/java/javase/downloads/jdk8-downloads-2133151.html}{Downlaod}
            \item 
            \href {https://netbeans.org/community/releases/81/install.html}{Installation steps}
        \end{itemize}
        \item \textbf{\large Software and code:}
        \begin{itemize}
        \item Code : 
        \href{https://github.com/eYSIP-2016/GUI-Testing-FirebirdV/tree/Firebird-GUI-Testing--Jatin}{Github link} 
        \end{itemize}
	
	\item \textbf{\large Use and demo:}
    \begin{figure}[h]
	    \begin{center}
		    \includegraphics[scale=0.60]{GUI.png}
	    \end{center}
    \end{figure}
    This is a GUI Application to test various components of Firebird V robot. So the various sections in this GUI tests various components of Firebird V.\\
    \newpage
    \head{Use the following steps to test the Firebird V : }
    \begin{enumerate}
    \item Firstly select the COM Port from the COM port section of the GUI and then browse the FirebirdFirmware(.hex file ) and click on the program buuton.
    \item Now, click on the connect button to make the connection to the robot.
    \item After connecting to the bot use all the sections of the GUI to test the Firebird V robot.
    \item In the Buzzer section, click on the ON button to turn on the buzzer and the buzzer beeps till the user turns off the buzzer by clicking on the button.
    \item Motion control section is used to control the forward, backward, right, left and stop motion of the bot using the buttons.
    \item Velocity control section is used to control the velocity of left and right motor. Set velocity using the slider or write in the text field and velocity can vary from 0 to 255. After setting the values, click on the Set button, a message would be seen that the velocity of both the motors have been set. Now the motion will be controlled by the velocity set by the user. On clicking the Reset button velocity of both the motors will be set to initial value i.e 255.
    \item Servo motor section is used to test the servo motors. Set the angle using slider or text field to rotate servo motor by that angle. User ca set the angle from 0 to 180 degrees as servo motor can rotate from 0 to 180 degrees.
    \item Movement and rotation section of GUI checks the position encoders. Rotate the robot left and right by specified angle giving the value of angle in text field. Also move the forward and backward by specified distance by giving distance in millimeter in text field.
    \item Print on LCD section prints one character on LCD. Enter the row, column and character to be printed in the text box and click on print button. This will print the character on the specified row and column. The LCD has two rows and 16 columns so user can enter rows from 1 to 2 and column from 1 to 16.
    \item Bar graph LED section is used to test the Bar Graph LED of the Firebird V robot. Enter the LED number to test in the text box and click on Glow button. It will glow the respective LED for 2 seconds. LED 8 is the top most LED and as we move down LED number decreases up to 1.
    \item Battery Voltage section gives the reading of battery voltage. Its maximum value is 12V.
    \item White line sensor section tests the white line sensors. It gives the value of left, center and right white line sensor. It can be used for to decide threshold value for line following. The lower readings are for white surface and higher readings are for black surface. 
    \item IR sensor section gives the reading of IR Sensors. IR Sensors 6, 7 and 8 works only when jumper 4 is connected. The readings of IR sensors vary from 0 to 255.
    \item Distance sensor section section gives the reading of the Distance sensor connected. The distance which is connected gives the correct reading all the other readings are garbage value. The readings of distance sensors vary from 0 mm to 800 mm.
    \item After testing all the components of the Firebird V robot click on the disconnect button in the COM Port section to close the connection.
    
\end{enumerate}


\newpage
\item \textbf{\large Future Work:}
 \begin{itemize}
	 	\item Print string consisting of more than one character on LCD. 
	 	\item Make a separate tab in GUI window which will be like the terminal window where whatever we read from the input stream and whatever we write on the outputstream of serial port is displayed.
	 	\item Auto-Testing of the Firebird V Robot can be implemented in the existing application.
	 \end{itemize} 

\item \textbf{\large Bug report and Challenges:}
\begin{itemize}
    \item Reading of different sensor values is not fast.
	\item Text boxes are taking values which are not acceptable. Like text boxes should not accept the values which are not range, also text boxes which requires only numerical value should not accept any other character.
\end{itemize} 

\end{enumerate} 
\newpage
\section*{References}
\begin{itemize}
    \item \href {http://www.javatpoint.com/java-tutorial}{http://www.javatpoint.com/java-tutorial}
	\item \href {http://docs.oracle.com/}{http://docs.oracle.com/}
	\item \href {http://stackoverflow.com/}{http://stackoverflow.com/}
	\item \href {http://www.miglayout.com/whitepaper.html}{http://www.miglayout.com/whitepaper.html}
	\item \href {http://www.miglayout.com/QuickStart.pdf}{http://www.miglayout.com/QuickStart.pdf}
	\item \href {http://rxtx.qbang.org/}{http://rxtx.qbang.org/}
\end{itemize}




\end{document}

